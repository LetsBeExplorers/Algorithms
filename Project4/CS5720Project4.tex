\documentclass{article}
\usepackage{graphicx}
\usepackage{booktabs}
\usepackage{geometry}
\usepackage{amsmath}

\geometry{margin=1in}

\title{CS5720 Project 4: Minimum Spanning Tree Analysis}
\author{Rachel Koch}
\date{\today}

\begin{document}

\maketitle

\section*{Attributions}

ChatGPT file changes: \textit{cs5720-proj4.py}
\begin{quote}
    Removed classify\_graph\_type() function \\
    Added graphing of vertices/edges to determine density \\
    Added graphs showing max number of edges for density \\
    Prim's algorithm implementation doesn't use heap anymore \\
    Results get sorted by graph name in the dataframe \\
    Removed graphing of the MST weights \\
    Kruskal's/Prim's runtimes are plotted separately \\
    Added V-squared line to Prim's plots \\
    Added ElogE line on Kruskal's plots
\end{quote}
ChatGPT file changes: \textit{Project4-GPT-FA24.tex}
\begin{quote}
    Changed the name of the file \\
    Added nodes and edges for all the graphs \\
    Removed density equation in Deliverable 2 \\
    Added vertex/edge density graphs to Deliverable 2 \\
    Added table showing MST weights to Deliverable 3 \\
    New graphs for Deliverable 4 \\
    Prim's and Kruskal's graphs are separate \\
    Added estimation/discussion of runtimes
\end{quote}


\section*{Introduction}
This report presents the analysis of 30 graphs categorized into three types. The tasks include:
\begin{itemize}
    \item Counting nodes and edges.
    \item Classifying graphs as sparse or dense.
    \item Computing the Minimum Spanning Tree (MST) using Prim's and Kruskal's algorithms.
    \item Empirically analyzing and comparing algorithm performance.
\end{itemize}

\newpage
\section*{Deliverable 1: Node and Edge Counts}
The following table summarizes the number of nodes and edges for each graph:
\vspace{.5cm}
\begin{table}[h!]
\centering
\begin{tabular}{lrr}
\toprule
\textbf{Graph} & \textbf{Nodes} & \textbf{Edges} \\
\midrule
Type 1 - t1\_graph\_0.csv & 10 & 17 \\
Type 1 - t1\_graph\_1.csv & 20 & 106 \\
Type 1 - t1\_graph\_2.csv & 30 & 299 \\
Type 1 - t1\_graph\_3.csv & 40 & 584 \\
Type 1 - t1\_graph\_4.csv & 50 & 971 \\
Type 1 - t1\_graph\_5.csv & 60 & 1447 \\
Type 1 - t1\_graph\_6.csv & 70 & 2029 \\
Type 1 - t1\_graph\_7.csv & 80 & 2729 \\
Type 1 - t1\_graph\_8.csv & 90 & 3511 \\
Type 1 - t1\_graph\_9.csv & 100 & 4384 \\
\hline
Type 2 - t2\_graph\_0.csv & 10 & 45 \\
Type 2 - t2\_graph\_1.csv & 20 & 172 \\
Type 2 - t2\_graph\_2.csv & 30 & 336 \\
Type 2 - t2\_graph\_3.csv & 40 & 505 \\
Type 2 - t2\_graph\_4.csv & 50 & 683 \\
Type 2 - t2\_graph\_5.csv & 60 & 898 \\
Type 2 - t2\_graph\_6.csv & 70 & 1067 \\
Type 2 - t2\_graph\_7.csv & 80 & 1296 \\
Type 2 - t2\_graph\_8.csv & 90 & 1504 \\
Type 2 - t2\_graph\_9.csv & 100 & 1675 \\
\hline
Type 3 - t3\_graph\_0.csv & 10 & 17 \\
Type 3 - t3\_graph\_1.csv & 20 & 40 \\
Type 3 - t3\_graph\_2.csv & 30 & 56 \\
Type 3 - t3\_graph\_3.csv & 40 & 76 \\
Type 3 - t3\_graph\_4.csv & 50 & 96 \\
Type 3 - t3\_graph\_5.csv & 60 & 114 \\
Type 3 - t3\_graph\_6.csv & 70 & 139 \\
Type 3 - t3\_graph\_7.csv & 80 & 158 \\
Type 3 - t3\_graph\_8.csv & 90 & 174 \\
Type 3 - t3\_graph\_9.csv & 100 & 197 \\
\bottomrule
\end{tabular}
\caption{Node and edge counts for all graphs.}
\label{tab:node_edge_counts}
\end{table}

\section*{Deliverable 2: Graph Sparsity or Density}
The classification of each graph type as sparse or dense is determined based on edge density.
Results indicate:
\begin{itemize}
    \item \textbf{Type 1}: Dense [\textit{See Figures 1 and 2} - not linear, number of edges approaches/mirrors maximum]
    \item \textbf{Type 2}: Sparse [\textit{See Figures 3 and 4} - linear relationship, does not approach max number of edges]
    \item \textbf{Type 3}: Sparse [\textit{See Figures 4 and 5} - linear relationship, does not approach max number of edges]
\end{itemize}

\newpage

\begin{figure}[h!]
\centering
\includegraphics[width=0.8\textwidth]{edge_vertex_relationship_type-1.png}
\caption{Relationship Between Number of Edges and Vertices for Type 1 Graphs.}
\end{figure}

\begin{figure}[h!]
\centering
\includegraphics[width=0.8\textwidth]{edge_vertex_max_relationship_type-1.png}
\caption{Edge/Vertex Relationship as Compared to the Max Number of Edges for Type 1 Graphs.}
\end{figure}

\newpage

\begin{figure}[h!]
\centering
\includegraphics[width=0.8\textwidth]{edge_vertex_relationship_type-2.png}
\caption{Relationship Between Number of Edges and Vertices for Type 2 Graphs.}
\end{figure}

\begin{figure}[h!]
\centering
\includegraphics[width=0.8\textwidth]{edge_vertex_max_relationship_type-2.png}
\caption{Edge/Vertex Relationship as Compared to the Max Number of Edges for Type 2 Graphs.}
\end{figure}

\newpage

\begin{figure}[h!]
\centering
\includegraphics[width=0.8\textwidth]{edge_vertex_relationship_type-3.png}
\caption{Relationship Between Number of Edges and Vertices for Type 3 Graphs.}
\end{figure}

\begin{figure}[h!]
\centering
\includegraphics[width=0.8\textwidth]{edge_vertex_max_relationship_type-3.png}
\caption{Edge/Vertex Relationship as Compared to the Max Number of Edges for Type 3 Graphs.}
\end{figure}

\newpage
\section*{Deliverable 3: MST Weights}
The total weights of the MSTs computed using Prim's and Kruskal's algorithms are summarized below. 

\vspace{.5cm}
\begin{table}[h!]
\centering
\begin{tabular}{lrr}
\toprule
\textbf{Graph} & \textbf{MST} \\
\midrule
Type 1 - t1\_graph\_0.csv & 54 \\
Type 1 - t1\_graph\_1.csv & 105 \\
Type 1 - t1\_graph\_2.csv & 109 \\
Type 1 - t1\_graph\_3.csv & 129 \\
Type 1 - t1\_graph\_4.csv & 140 \\
Type 1 - t1\_graph\_5.csv & 173 \\
Type 1 - t1\_graph\_6.csv & 180 \\
Type 1 - t1\_graph\_7.csv & 213 \\
Type 1 - t1\_graph\_8.csv & 244 \\
Type 1 - t1\_graph\_9.csv & 247 \\
\hline
Type 2 - t2\_graph\_0.csv & 15 \\
Type 2 - t2\_graph\_1.csv & 23 \\
Type 2 - t2\_graph\_2.csv & 30 \\
Type 2 - t2\_graph\_3.csv & 42 \\
Type 2 - t2\_graph\_4.csv & 53 \\
Type 2 - t2\_graph\_5.csv & 60 \\
Type 2 - t2\_graph\_6.csv & 73 \\
Type 2 - t2\_graph\_7.csv & 83 \\
Type 2 - t2\_graph\_8.csv & 91 \\
Type 2 - t2\_graph\_9.csv & 101 \\
\hline
Type 3 - t3\_graph\_0.csv & 32 \\
Type 3 - t3\_graph\_1.csv & 50 \\
Type 3 - t3\_graph\_2.csv & 99 \\
Type 3 - t3\_graph\_3.csv & 97 \\
Type 3 - t3\_graph\_4.csv & 140 \\
Type 3 - t3\_graph\_5.csv & 213 \\
Type 3 - t3\_graph\_6.csv & 182 \\
Type 3 - t3\_graph\_7.csv & 261 \\
Type 3 - t3\_graph\_8.csv & 266 \\
Type 3 - t3\_graph\_9.csv & 320 \\
\bottomrule
\end{tabular}
\caption{MSTs for all graphs.}
\label{tab:mst_graphs}
\end{table}

\newpage
\section*{Deliverable 4: Timing Analysis}
The empirical timing results for Prim's and Kruskal's algorithms are shown below. 

\subsection*{Prim's Algorithm}

\begin{figure}[h!]
\centering
\includegraphics[width=0.8\textwidth]{timing-analysis-Prim-type-1.png}
\caption{Prim's Timing Analysis for Type 1 Graphs.}
\end{figure}

The type-1 graph results in Figure 7 appear exponential - $\Theta(e^{|V|})$ - which exceeds the quadratic runtime found in Homework 8 - ($\Theta(|V|^2)$). This was the densest set of graphs, so the runtimes overall were much higher than for types-2 and 3. It's possible that given more data points, this plot would approach quadratic - but it's hard to tell from the provided ten graphs. There is also a possibility that my computer cache is affecting the length of later operations, but it would require more in-depth analysis.  

\begin{figure}[h!]
\centering
\includegraphics[width=0.8\textwidth]{timing-analysis-Prim-type-2.png}
\caption{Prim's Timing Analysis for Type 2 Graphs.}
\end{figure}

\begin{figure}[h!]
\centering
\includegraphics[width=0.8\textwidth]{timing-analysis-Prim-type-3.png}
\caption{Prim's Timing Analysis for Type 3 Graphs.}
\end{figure}

The type-2 results in Figure 8 and the type-3 results in Figure 9 both appear much more quadratic - ($\Theta(|V|^2)$). This matches the runtime as found in Homework 8 and confirms what we know about Prim's algorithm. 

A scaled version of $|V|^2$ was included on Figures 7, 8, and 9 for comparison (red line). These lines show that in Figures 7, 8, and 9 the time complexity of our algorithm either meets or exceeds $\Theta(|V|^2)$.

\newpage
\subsection*{Kruskal's Algorithm}

\begin{figure}[h!]
\centering
\includegraphics[width=0.77\textwidth]{timing-analysis-Kruskal-type-1.png}
\caption{Kruskal's Timing Analysis for Type 1 Graphs.}
\end{figure}

\begin{figure}[h!]
\centering
\includegraphics[width=0.77\textwidth]{timing-analysis-Kruskal-type-2.png}
\caption{Kruskal's Timing Analysis for Type 2 Graphs.}
\end{figure}

\newpage
\begin{figure}[h!]
\centering
\includegraphics[width=0.8\textwidth]{timing-analysis-Kruskal-type-3.png}
\caption{Kruskal's Timing Analysis for Type 3 Graphs.}
\end{figure}

Figures 10 and 11 show that Kruskal's algorithm was not only faster than Prim's, but that it clearly resulted in the expected time complexity of $\Theta(E\log E)$ for types-1 and 2. 

Our type-3 graphs in Figure 12 appear to have quadratic time complexity - $\Theta(|V|^2)$. This does not match what was found in Homework 8 for Kruskal's algorithm in general, but it does still exceed $\Theta(E\log E)$ as requested.

A scaled version of $E\log E$ was included on Figures 10, 11, and 12 for comparison (green line). These lines show that the time complexity of our implementation either met or exceeded $\Theta(E\log E)$.

\subsection*{Estimates}
Time Complexity estimates are summarized below:

\vspace{.5cm}
\begin{table}[h!]
\centering
\begin{tabular}{lrr}
\toprule
\textbf{Graph Type} & \textbf{Prim's} & \textbf{Kruskal's} \\
\midrule
Type 1 & $\Theta(e^{|V|})$ & $\Theta(E\log E)$ \\
\hline
Type 2 & $\Theta(|V|^2)$ & $\Theta(E\log E)$ \\
\hline
Type 3 & $\Theta(|V|^2)$ & $\Theta(|V|^2)$ \\
\bottomrule
\end{tabular}
\caption{Time Complexity Estimates for all graphs.}
\label{tab:tc_graphs}
\end{table}

\section*{Conclusion}
This project demonstrated the implementation and analysis of MST algorithms across graph types. The results mostly aligned with theoretical expectations, and empirical findings provided deeper insights into algorithm performance.

\end{document}
